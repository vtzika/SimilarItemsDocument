\chapter{Methods}

In this section, we describe the methods used in the approach we follow creating queries to 

\section{Query modeling}

A query is the representation of a user's information need. Enhancing the query by changing or expanding it is called query modeling. Query modeling creation involves the preprocessing step and the identification of tokens. The right query model is the most important element that will affect the resulting classified list. There are several ways of query modeling. We explore three different kinds: a) exploiting the previously visited classified structure b) extracting discriminative terms based on the log likelihood ratio and c) expanding queries by using pseudo relevance feedback.

\subsection{Exploiting the classified structure}

The previously visited classified is an indication of the user's interest. Thus, important information can be extracted from this visited or source classified for creating query models. Classifieds typically consist of title, description and category. Some classifieds also consist of attributes specific to the category. We believe that each field has different information and using this information can enhance our results. The title summarizes the contents of the classified. It only consists of a few important words and we assume that using these words in the query model will retrieve highly relevant classifieds. Description is a more detailed representation of the classified. We assume that using words from the description in the query model will retrieve a broader result than using the title's words. Also, description contains more noise than the title's contents due to the amount of words it contains and we believe that we will retrieve a lot of irrelevant documents in the query model by using only description. From the other side, combining title words with description words will give a boost in the words that are present in both fields and it will expand the result list with classifieds matching the description words that don't exist in the title. Category is the category that classified is part of and is very generic. Using category words in our query model will retrieve a broad list of classifieds that don't cover our scope to find the most relevant classifieds first. We assume though that combining the attributes with categories can retrieve relevant classifieds. Furthermore, we believe that combining all these fields together will result the most relevant classifieds list. Category and attributes fields consist of different information than description and title and combining them means that we have a full descriptive representation of our visited classified which increases the chances to retrieve a highly relevant classifieds list. The combination of content fields that are mapped to queries and multiple query models are presented in table \ref{table:QueryFields}.


\begin{table}[H]
\begin{center}
\scriptsize
\caption{A summary of the query models we consider is presented. We mark the classifieds fields we use on the term extraction and weather we use log likelihood ratio (LLR), pseudo relevance feedback (PRF), stemming or not.}
\label{table:QueryFields}
\begin{tabular}{lccccr}
\midrule
Name & Fields & Stemming & LLR & PRF \\
\midrule
T & title & no & no & no \\
T-LLR & title & no & yes & no \\
T stemming & title & yes & no & no \\
T-LLR stemming & title & yes & yes & no \\
T+D  & title, description & no & no & no \\
T+D+A+C & title, description, category, attributes & no & no & no \\
A+C & attributes, category & no & no & no \\
T+D+A+C-LLR & title, description, category, attributes & no & yes & no \\
T+D+A+C-Pseudo & title, description, category, attributes & no & no & yes \\
T-Pseudo & title & no & no & yes\\
\bottomrule
\end{tabular}
\end{center}
\end{table}

\subsection{Pseudo relevance feedback}

The general idea behind relevance feedback is to take feedback from the top results that are initially retrieved from a given query. Although, pseudo relevance feedback improves the efficiency of the system, it is dependable to the initial query since it assumes that the top k results are relevant. However, through query expansion, some relevant documents that are missed in the initial round can then be retrieved to improve the overall performance. Clearly, the effect of this method strongly relies on the quality of the selected expansion terms but we assume that our initially retrieved classifieds list will be improved. 

\subsection{Log likelihood ratio}

Term extraction is a key component in query modeling i.e. which are the right terms for retrieving classifieds that address the user's information need? A few methods exist to retrieve discriminative terms like idf or Log likelihood ratio. We consider the log likelihood ratio (LLR) and we create query models by using terms extracted with this method from the visited classified.

``A likelihood ratio test is a statistical test used to compare the fit of two models, one of which (the null model) is a special case of the other (the alternative model). The test is based on the likelihood ratio, which expresses how many times more likely the data are under one model than the other.''

As it is described in \cite{RaysonGarside} we can use LLR to compare corpora and find the terms of a corpus that are more characteristic. There are two main types of corpus comparison:

\begin{list}{.}{}
\item Comparison of a sample corpus to a larger corpus (normative)
\item Comparison of two (roughly) equal sized corpora
\end{list}

These two main types of comparison can be extended to the comparison of more than two corpora. For example, we may compare one normative corpus to several smaller corpora at the same time, or compare three or more equal sized corpora to each other. In general, however, this makes the results more difficult to interpret.

This first type of comparison is intended to discover features in the sample corpus with significantly different usage (i.e. frequency) to that found in ``general'' language. While second type aims to discover features in the corpora that distinguish one from another. In our case, the first type is more appropriate since we need to find a way to distinguish a model for a classified against a large corpus that will give us enough feedback for every word in the classified. We refer to the larger corpus as a ``normative'' corpus since it provides a text norm (or standard) against which we can compare.

The representativeness of the big corpus needs to be considered when comparing two corpora. It should contain samples of all major text types and if possible in some way proportional to their usage in the natural writing of a classified in case we want features (in our case frequencies) to make sense . In the case of classifieds created by users, we need a corpora with a data set of classifieds really created by users and big to contain almost all different words a user will write in his classified \cite{RaysonGarside}.

We can create query models with the use of LLR using the top k words with the biggest LLR number. This means that we have to calculate LL for every word in a given classified.

The method that we have to follow is the following:
Given a visited classified as null corpora and a big dataset of classifieds as normative corpora that we wish to compare with, we produce a frequency list. This would be a word frequency list. For each word in the first frequency list we calculate the log- likelihood statistic. This is performed by constructing a contingency table see table \ref{table:contigency}.


\begin{table}[H]
\begin{center}
\footnotesize
\caption{Contingency table for Log likelihood calculation.}
\label{table:contigency}
\begin{tabular}{lccr}
\toprule
&  First Corpus & Second Corpus & Total \\
\midrule
Frequency of word & a &	b & a+b \\
Frequency of other words & c-a & d-b & c+d-a-b \\
Total & c & d & c+d \\
\bottomrule
\end{tabular}
\end{center}
\end{table}

Then, we need to calculate the expected values (E) according to the following formula:

\begin{equation}
E_i= \frac{N_{i}\sum_{i}{O_{i}}}{\sum_{i}{N_{i}}}
\end{equation}

The calculation for the expected values takes into account the size of the two corpora, so we do not need to normalize the figures before applying the formula. We can then calculate the log-likelihood value according to this formula:

\begin{equation}
-2ln\lambda=2\sum_{i}{O_{i}}{ln \frac{O_{i}}{E_{i}}}
\end{equation}

This equates to calculating log-likelihood LL as follows:

\begin{equation}
LL =2*((a*ln \frac{a}{E1}) + (b*ln \frac{b}{E2}))
\end{equation}

The word frequency list is then sorted by the resulting LL values. This gives the effect of placing the largest LL value at the top of the list representing the word which has the most significant relative frequency difference between the two corpora. In this way, we can find the words most indicative (or characteristic) of one corpus, as compared to the other corpus, at the top of the list \cite{RaysonGarside}.


\section{Retrieval models}

A retrieval model takes a query and a classified as input and identifies a measure of relevance between the query and the classified. Different retrieval models have different retrieval strategies thus resulted documents differs as well.

Retrieval model or ranking function used by search engines mostly. A search engine except of finding the relevant document, has to rank and order them by relevance. This is typically done by assigning a numerical score to each document based on a ranking function, which incorporates features of the document, the query, and the overall document collection.

The study of retrieval models is central to information retrieval. Many different retrieval models have been proposed and tested, including vector space models, probabilistic models and logic-based model.

The ranking functions-retrieval models that we will use for this project are the following:
\begin{list}{.}{}
\item TfIdf
\item Okapi BM25
\item Probabilistic Language Model
\end{list}

\subsection{TfIdf}

TfIdf (term frequency-inverse document frequency) used as term weighting factor in information retrieval. This retrieval method ranks documents based on characteristic terms of a document. Characteristic terms for a document are those who only frequently appear in the possible relevant document while infrequently in the rest documents of data collection \cite{ShouningSujuanYan}.

TF is the frequency and idf is its inverse document frequency. Term frequency in a given document is the number of times a given term appears in that document. The inverse document frequency is a measure of whether the term is common or rare across all documents.

TfIdf  is calculated as:
\begin{equation}
	TfIdf = tf * idf
\end{equation}

\begin{equation}
	idf = log \frac{ d }{ d_t}
\end{equation}

Where d is the total number of documents in the collection, $d_t$ is the  total number of documents where term t occurs. However if the term t does not occur in the document collection idf, then dt will be equal to zero. Therefore the formula is adjusted to 1+dt. The advantage of TfIdf is that it tends to filter out common terms. When a document has high term frequency while the term appears rarely in the whole collection of documents then it has high weight in TfIdf scoring.


\subsection{Okapi BM25}

In information retrieval, Okapi Best match 25 (BM25)  is a ranking function used by search engines to rank matching documents according to their relevance to a given search query. Okapi ranking function is based on the probabilistic retrieval framework. It makes an estimation of the probability of finding if a document dj is relevant to a query q. Three factors affects Okapi's score. First is the query terms frequency, second is the inverted frequency of query terms and finally, the length of the document. With this way, it scores higher a short document that mention all query terms.

Given a query , containing keywords , the BM25 score of a document is:
\begin{equation}
BM25(d_j,q_i:N) = \frac{ ΣIdf(q_i)*Tf(q_i, d_j)*(k + 1) } { ( tf(q_i, dj)+k* (1-b+(b*|d_j|/L))) }
\end{equation}

Where N is the total number of documents, tf($q_i$, $d_j$) is the frequency of $q_i$ word in $d_j$ document and idf($q_i$) is the inverse document frequency of word given by:

\begin{equation}
idf(q_i) = log \frac{ N-DF(q_i) + 0.5 } { DF(q_i) +0.5 }
\end{equation}

Where $d_j$ is the length of document $d_j$ in words, L is the average document length in the corpus.

\subsection{Language modeling}

Language Modeling estimates the probability distribution of linguistic units such as words, sentences, queries, utterances, or even complete documents. The probability distribution itself is referred to as a language model \cite{CroftLafferty}. Given the query q and the user U, we want to find the most probable documents. That is, we want to rank the documents by p($d|q$, U ). Using Bayes' theorem:

\begin{equation}
p(d|q, U ) = \frac{p(d|U )p(q|d, U )} {p(q|U)}
\end{equation}

For the purposes of ranking, we can ignore the denominator and define the relevance of a document as:

\begin{equation}
pq (d) = p(d|U ) * p(q|d, U )
\end{equation}

The query likelihood p($q|d$) is calculated by assuming that the query terms are independent, and then multiplying the probabilities for the individual terms. If the query q = ($q_1 q_2 \ldots q_m$ ) , then:

\[p(q|d) =\prod_{i->1}^{m} q_i\]


Furthermore, suppose that we have the query ``This is a great book for retrieval and evaluation in IR'' created by the description of a book and also we have as candidate document with description ``This is a book for evaluation in IR''. The candidate document does not contain the query word ``retrieval''. Now, if we estimate p($retrieval|d$) ,then this probability will be zero and the query likelihood will vanish. Thus, the language model for a document has to distribute some probability mass among words that are not in the document too. This task is called smoothing  \cite{ZhaiLaferty}. Dirichlet smoothing is used to solve the zero probability and data sparseness problems.

\begin{equation}
p(q|d) =\frac{tf + m * p(q|C) }{|D| + m}
\end{equation}


\section{Diversification}

\subsubsection{Maximal marginal relevance}
Diversification is implemented to provide of more diversified result set. Maximal Marginal Relevance (MMR) is a diversification method aims to re-rank the result set selecting the highest combination of a similarity score between classifieds and their initial rank. It is defined as:


\begin{equation}
MMR\overset{def}{=}\operatorname*{Arg\, \max}_{D_{i} \in R\ \setminus S}[\lambda(Sim_{1}(Di,Q)-(1-\lambda)\underset{{D_{j}\in S}}{\max}Sim_{2}(D_{i},D_{j})]
\end{equation}

where:
R : Rank list of documents retrieved by an IR system

S : is the subset of documents in R already selected

Sim1 : is the similarity between documents and a query

Sim2 : the similarity between the documents

$\lambda$ : 0.5 because we want to give the same weight to ranking order and diversity


\subsection{Maximal marginal relevance alternative 1}

We introduce a flavor of MMR called Maximal Marginal Relevance alternative (MMRalt). It is a diversification method aims to re-rank the result set selecting the highest combination of a similarity score between pairs of classifieds and their initial rank. The main difference with MMR is that MMRalt will result a list with consecutive diversified pairs of classifieds. 


\bigskip

\begin{equation}
MMRAlt\overset{def}{=}\operatorname*{Arg \, \max}_{D_{i}\in R\ \setminus S}[\lambda(Sim_{1}(Di,Q)-(1-\lambda) Sim_{2}(D_{i},D_{s})]
\end{equation}

R : Rank list of documents retrieved by an IR system

S : is the subset of documents in R already selected

s : is the previous document selected

Sim1 : is the similarity between documents and a query

Sim2 : the similarity between the documents

$\lambda$ : 0.5 because we want to give the same weight to ranking order and diversity






\bigskip

\begin{algorithm}[H]
 \KwData{Given the initially ranked classifieds list}
 \KwResult{A diversified ranked classifieds list}
 initialization\;
 \While{we still have non selected classifieds in the given list}{
	  choose the first one and expose it
	  $already displayed list \gets classified in the first ranked position$
	  calculate the MMR of documentX using as cosine similarity the similarity between X and already displayed
	  expose the classifiedZ with the max(MMR) as the next one
	  $already displayed \gets classifiedZ$
	  remove the classifiedZ of the given list
 }
 \caption{MMRalt algorithm.}
\end{algorithm}

\subsection{Maximal marginal relevance alternative 2}

We introduce an alternative flavor of the MMR algorithm, the Maximal Marginal Relevance alternative 2 (MMRalt2). This diversification method aims to re-rank the result set selecting the highest combination of a similarity score between classifieds and their initial rank. The difference between MMR and MMRalt2 is that this technique does not take into account the maximum cosine similarity between a classifies and the displayed classifieds. Instead it takes into account the average MMRalt2 score between a document and the previously displayed classifieds. The difference between MMRalt and MMRalt2 is that the later aims to have the entire result list diversified. The hypothesis here is that taking into account the average MMR score between all the previously ranked classifieds and the next proposed classified is a better indication of similarity than the MAX cosine similarity. 
\bigskip

\begin{equation}
MMRAlt2\overset{def}{=}\operatorname*{Arg \, \max}_{D_{i}\in R,s}
\end{equation}


R :Rank list of documents retrieved by an IR system

S :is the previous document selected

\bigskip

\begin{algorithm}[H]
 \KwData{Given the initially ranked classifieds list}
 \KwResult{A diversified ranked classifieds list}
 initialization\;
 \While{we still have non selected classifieds in the given list}{
	  choose the first one and expose it
	  $doc_selected \gets classified in the first ranked position$
	  $MMRx \gets AVG(MMR_doc_selected)$
	  expose the classifiedZ with the max(MMR) as the next one
	  $already displayed \gets classifiedZ$
	  remove the classifiedZ of the given list
 }
 \caption{MMRalt2 algorithm.}
\end{algorithm}





\subsection{Maximal marginal relevance within a range of four classifieds}

This proposed algorithm which we called MMRalt2last4 is a diversification method which aims to re-rank the result set by selecting the highest combination of a similarity score between the last four re-ranked classifieds and their initial rank. The main difference between the other approaches is that this technique takes into account only the last four selected documents instead of all of them. The resulted list with similar classifieds that Marktplaats decided to expose will be in pages consisting of four classifieds. With our proposed way, each page will always consist of diversified results and the initial rank will not be affected as much as the other approaches. 

\bigskip

\begin{algorithm}[H]
 \KwData{Given the initially ranked classifieds list}
 \KwResult{A diversified ranked classifieds list}
 initialization\;
 $doc_selected \gets classified in the first ranked position$
	  
 \While{we still have non selected classifieds in the given list}{
	  choose the first one and expose it
	  $doc_selected \gets classified in the first ranked position$
	  $MMRx \gets AVG(MMR_doc_selected_last_four)$
	  expose the classifiedZ with the max(MMR) as the next one
	  $already displayed \gets classifiedZ$
	  remove the classifiedZ of the given list
 }
 \caption{MMRalt2last4 algorithm.}
\end{algorithm}



\section{Late data fusion}

Based on \cite{Wilkins}, data fusion is the process of integration of multiple data and knowledge representing the same real-world object into a consistent, accurate, and useful representation. There are two approaches for the combination of data known as early fusion or late fusion. Early fusion is the combination of data prior to indexing. Thus the data aggregated and then retrieval model use this aggregated data as input. While late fusion assumes each source of data has associated with it some form of a ranking function, each of which can be independently queried. Once each source has been queried, the outputs of each of these queries can be aggregated together to form a final response to the initial query.

Since different retrieval results can generate quite different ranges of similarity values, a normalization
method should be applied to each retrieval result. Normalization controls the ranges of similarity values that retrieval systems generate. Hence, in order to align both the lower bounds of similarity values and the upper, we normalize each similarity value
by the maximum and minimum actually seen in a retrieval result as follows:

\begin{equation}
normalized score(x) = \frac{x-min}{max-min}
\end{equation}

Where min is the minimum value for the ranked list and max is the maximum value for the ranked list.After the normalization of the score we merged all of the ranked lists. We consider the following late data fusion methods explained on table \ref{table:LDFmethods}:
\bigskip

\begin{table}[H]
\begin{center}
\caption{Late data fusion methods we consider.}
\label{table:LDFmethods}
\begin{tabular}{lcc}
\midrule
Name &   & Explanation \\
\midrule
combMAX &   & Maximum of individual scores \\
combMIN &   & Minimum of individual scores \\
combSUM &   & Sum of individual scores \\
combANZ &   & combSUM / number of non zero scores \\
combMNZ &   & combSUM * number of non zero scores \\
WcombSUM &   & weighted sum of individual scores \\
WcombMNZ &   & WcombSUM * count of non zero results \\
WcombWW &   & WcombSUM * sum of individual weights \\
\bottomrule
\end{tabular}
\end{center}
\end{table}

Then, we diversify all of the merged ranked lists using the diversification methods described above.

Next chapter will cover the explanation of our experimental design using the methods described in this chapter.