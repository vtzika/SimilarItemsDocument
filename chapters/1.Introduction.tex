\abstract

Our goal in this thesis is to find an optimal solution for similar classifieds list creation. Similar classifieds list consists of proposed ads recommended to the users to help them find what they need. We propose to recommend a similar classifieds list based on user's previous interest. Therefore, the evaluation of the similar classified list will be based on the relevance of each classified with the last visited classified. Consequently, we derive queries as indication of user's interest from the previous visited classified. These queries are then used to retrieve similar classifieds from a classifieds index, resulting in multiple ranked lists. Also, we diversify these lists to investigate the impact on precision. We merge these initial lists as well as the diversified lists using data fusion techniques because usually merging multiple lists together results to a better than the individual systems precision. To evaluate our experiments, we use data from an entrepreneurial database with users created classifieds. We show that using all the information we have available from the visited classified produces a result list with high precision. Also, we prove that fusion techniques are not improving the precision of individual systems. However, the fusion of diversified result lists has the highest precision. Furthermore, we propose three alternative diversification methods that are not having any change on the precision. Finally we present our observations, we give future work ideas and we conclude this thesis.


\newpage
\tableofcontents


\chapter{Introduction}


Marktplaats is an e-commerce platform where users are able sell products or services by creating classifieds. Classified is a way for users to list their  items, services, or properties for sale without creating an auction-style or fixed price listing. One of Marktplaats main goals is to keep their users satisfied so they will improve their reputation and revenues. To keep users satisfied, they want to help users find the most relevant product or service as quickly as possible. Users are searching for available classifieds with a word or bag of words that represent the query. However, there words that can be interpreted with multiple ways and using this words in a query results ambiguity. Ambiguous ueries with short query length in a large amount of classifieds produces a big and broad list of proposed classifieds. For instance the word Java can be interpreted as Java island or Java programming language or Java coffee and all of their related classifieds will be retrieved by the search engine \cite{RoulSahay}. If users are provided with a very big broad list of classifieds, they will spend a lot of time to find what they want. Instead, providing him with a small and concrete list with classifieds will save him time. Taking into account that users are examining a classified (reading its contents) only when it appeals to them, we are attempting to solve that problem by retrieving and recommending to users a list of relevant classifieds based on the previously examined classified. A classified consists of content describing elements (fields) such as title, description and attributes that are created by common users. The previously visited classified field's contents are extra information that can help us retrieve a list with classifieds based on them.

Many companies had to come up with related solutions to problems like this. Google News groups news by story rather than presenting a raw listing of all articles. Also, they record every click or search that every user made and just below the ``Top Stories'' section users can see a section labeled ``Recommended for your email address'' along with three stories that are recommended to them based on their past click history. This way Google uses users history to predict their interest and give better recommendations \cite{AbhinandanGoogle}. Likewise, Amazon based on users past shopping history and site activity recommends books and other items likely to be of interest. Also, Amazon is mapping items to the list of similar items. For example, if there is a list with 3 items that a user is interested in, then this becomes an item list. If another user is interested in one of these items, then the rest of the items in the list will be recommended to him as well \cite{LindenJacobiBenson}. In the music industry, Schlitter and Falkowski use data from Last.fm and create user profiles based on the genres of the most listened artists \cite{SchlitterFalkowski}. They create communities based on the categorization and they recommend music to users based on the category of the community they belong to. For example, one community is hip hop and its members are recommended to listen to hip hop music \cite{OwenAnil}.

Previously described companies solve the problem in different ways which we could use as well. One of the options is to classify items in categories as done by Last.fm. However, trawling through hundreds or thousands of categories and subcategories of data is not an efficient method for finding information \cite{HatcherMcCandless}.  Another option is to use a recommendation algorithm based on other users history but then we have to deal also with new users that don't have any history to relate with others. Another constraint is that classifieds are not active for a long time, thus making recommendations based on user history is difficult. Offline solutions such as Amazon's which generates similar item lists periodically, is not useful with classifieds because of their short life span. We could also use the click logs to improve the ranking of the search results via the use of the so called learning to rank methods, however these methods rely on already optimal search algorithms that are currently lacking at Marktplaats. Our goal is first to explore the utility of several retrieval methods for retrieving relevant classifieds. In future work we will explore click models and learning to rank approaches.

The large amount of information a classified has is what we need to relate user’s behavior with better recommendations. However, to show to user relevant classifieds based on their interest requires finding the relevant classifieds based on their history and to relate the information with other classifieds. Recommending similar classifieds to users, involves retrieving the most important information of the classified which the users have already seen. Classifieds have this information either on their description or in their attributes. Extracting this knowledge from enormous amounts of data can be achieved with methods from the information retrieval area which is defined as ``the area of study concerned with searching for documents, for information within documents, and for metadata about documents.''\cite{SinghSingh}.

To implement this approach we extract terms of a classified that represent the information need of the user. This procedure is called query modeling. Then, given a retrieval strategy and a query, the search engine responds with a list with classifieds. The retrieval strategy is responsible to search in our indexed classifieds based on the input (query) and they will produce the list with similar classifieds (result list).

A good choice of the retrieval strategy or the query modeling will be proved once adequate lists are constructed based on the input that we gave. Different query models will result in a different result list. Different retrieval strategies will result in different result lists as well. The choice of the best query model and retrieval strategy are subjects of experimentation. The results the models give us will affect the performance and the accuracy of the system. A good performing system or retrieval strategy will be evaluated based on the precision of the system. That means that it will be affected by the number of relevant results retrieved by a system and the ordering (rank). System is the combination of the query modeling, retrieval strategy and all the systemic properties like how did we process the classified (e.g. removing noise words like 'the', 'or' etc.). Our work is implemented on a company, where data and different kind of users can help us to evaluate different query models.

Different query models will be created based on different combination of the fields. Also, we can extract discriminative terms of the visited classified to represent the information need. Furthermore, we can take feedback from a result list and create a new query that will result a new list which is the recommendation.

Retrieval strategies have existed since the early 90's and there is no need to create a new one. In this thesis we will examine three popular retrieval strategies, TfIdf, Okapi BM25F and LM. We will investigate which strategy will increase the performance of our systems.

In a large amount of classifieds it is possible that there are lots of possible redundant or containing partially or fully duplicative information. Our goal is to expose less classifieds with high potential to cover the information need of the user. Experiments will be conducted to find the best way to provide a diversified result lists instead of a list with lots of duplicated classifieds. Since user's information need are often ambiguous, we can give to the user more diversified results to increase the possibility that a classified will satisfy their information need.

To improve our results even further, we use late data fusion techniques \cite{Belkin}. Merging the result lists that are retrieved by multiple query models to one new list improves the performance of individual systems. Experiments with multiple late fusion techniques are provided. We also use fusion methods on the diversified result lists to compare and see if there is any improvement.

To conclude, we derive multiple query models from a given classified, which are then used to retrieve similar classifieds from an index, resulting in multiple ranked lists. We then diversify these lists. Then we merge this initial lists as well as the diversified lists using data fusion techniques. Query models are created by exploiting the structure of the given classified and by discriminative terms of the classifieds context. Following we present the research questions we want to answer:

The research questions we aim to answer are the following:
\begin{enumerate}
\item Which query model is the best to improve the performance of title query model?
\item Which of the three (TfIdf, Okapi BM25F, LM) retrieval strategies is performing better?
\item Does the fusion of individual strategies improve the performance of the individuals query models?
\item Is the results of diversification affected if only the similarity with the previous displayed classified is taken into account?
\item Is the results of diversification affected if the average similarity of previous displayed docs is taken into account?
\item Is the results of diversification affected if only the similarity with the previous four displayed classifieds is taken into account?
\item Does the fusion of diversified systems improve the performance of the not diversified systems?
\end{enumerate}

The remainder of this thesis is organized as follows. In the second section we describe related work. Third section serves as introductory chapter to the field of information retrieval and explains important terms of the method that we follow. Section four presents the experimental framework. Fifth section offers experimental results. Section six analyzes the results and seven section concludes this thesis and discusses future directions.




